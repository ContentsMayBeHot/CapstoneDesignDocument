\chapter{Conclusion}


\section{Context and Project Importance}
%-------------------------------------------%

This project demonstrates the viability of \textit{Rivals of Aether} as a candidate for future game-related AI research. It is entirely possible and even practical to use replays to build a robust dataset for supervised learning.

%-------------------------------------------%


\section{Limitations and Delimitations}
%-------------------------------------------%

The data processing pipeline constitutes the most significant portion of this project. The machine learning implementation is only a proof of concept, and the game agent's gaming competence should not be used to measure the success of the project.

%-------------------------------------------%


\section{Future Work}
%-------------------------------------------%

There are three major tasks remaining. The first is a refactor of the entire project, sans the machine learning program, into a fully-fledged Python library. Merging the projects together in this fashion will make the project simpler, overall, because all of the modules will be able to share resources with one another. The second task, which comes as part of the refactor, is improving the unit test coverage of the entire project. The third and final task will be improving the machine learning model. This will require significant experimentation. It is likely that the next iteration will take cues from Arnold by segregating predictions for movements from those for attacks \cite{Chaplot}.

%-------------------------------------------%